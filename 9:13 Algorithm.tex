\documentclass[12pt]{report}
\usepackage{amsmath}
\usepackage{color}
\title{Algorithm Class 1}
\author{Introduction}
\date{September 13, 2016}
%\parindent=0pt
\begin{document}
\maketitle
\setlength{\abovedisplayskip}{0pt}
\setlength{\belowdisplayskip}{0pt}

\chapter*{Introduction}

\section*{What's Algorithm?}
\begin{itemize}

\item Algorithm: A step-by-step procedure to solve problems.
\item Problem: Input $\to$ Output\\
ex: multiplication: a, b $\to$ a$\times$b\\

ex: sorting (Quick sort, Merge sort, Bubble sort)
\newline
\hspace*{1.3em} Implementation: computer program
\item Why study algorithm?

\begin{enumerate}
\item Performance determines feasibility.

\item Provide a language to talk about program behaviors.

\item Generalize to other resources.

\item Just for fun yeeeeee$\sim\sim$
\end{enumerate}
\item Example: Multiplication
\\
1234$\times$4321\\

\begin{equation}
\frac{
    \begin{array}[b]{r}
      1234\\
      \times \left 4321 \right
    \end{array}
  }
  {
  	\begin{array}[b]{r}
    \quad 1234\\
    2468\enspace\\
    \vdots ~ \quad
    \end{array}
  }
\end{equation}
Q: How many single-digit multiplications is required??
\\
A: T(n) = \# of single-digit multiplications required to multiply 2\\ n-digit
. ~T(n) = n$^2$
 \\[10pt]
Q: Can it be fewer?
\begin{description}
\item[Idea 1:]Use recursion\\
 n-digit number x = a\times 10$^\frac{n}{2}$ + b
\\
 a,b = $\frac{n}{2}$ - digit numbers
\\     
 ex: \\
 $1234 = 12 \times 10^2 + 34
\\$                       
 $4321 = 43\times 10^2 + 21
\\$                 
 xy = ac \times~ 10^n + (ad+bc)\times 10^$\frac{n}{2}$ + bd
\\                 
 compute xy $\iff$ compute ac, ad, bc, bd, shift digits, sum up
\\                               
 (xy = T(n) = 4T($\frac{n}{2}$)), T(n) = n^2)

\item[Idea 2:]Karatsuba

\begin{align*}
 xy &= ac\times 10^n + (ad+bc)\times 10^\frac{n}{2} + bd
\\                     
 &= ac\times 10^n + [(a+b)(c+d)-ac-bd]\times 10^\frac{n}{2} +bd
\\                
 T(n) &= 3T(\frac{n}{2}) \implies T(n) \propto n^{1.585}

\end{align*}

\item[Extensions:]Toom-Cook Scheme: n$^{1.465}$\\
Fast Fourier Transform: $n\times\log(n)\times\log\log(n)$
\end{description}
\item Example: sequence alignment\\
Input: 2 sequences\\
\hspace*{2.6em} (1)CGTTCAT ~~~~~(2)CGTTAC\\
Output: Similarity score\\
Alignment: \\
\hspace*{2em}CGTTCAT\\
\hspace*{2em}CGTT~~AC\\
\hspace*{5em}{\color{red}$\uparrow$\quad$\uparrow$\\
\hspace*{4em}gap\hspace{0.5em} mismatch\\}\\
Score: C\textsubscript{1}(\# of gaps) + C\textsubscript{2}(\# of mismatchs)\\
Target: Find the minimum score in all alignments
\begin{description}
\item[Idea 1:]brutal force\\
Try all alignment $\rightarrow$ pick the minimum score.\\
for length = 200$\sim$300, takes longer than the age of universe.
\item[Idea 2:]Greedy\\[0.8em]
CGTTCAT ~~~~~CGTTAC\\
CGTT~~AC ~~~~~CGTT~~CAT\\
\end{description}
\item Running time (Time complexity) of an algorithm\\
\# of steps: there's max and min number of steps.
\item Worst case complexity
\begin{itemize}
\item guarantee
\item used in the class unless it's specified
\end{itemize}
\item Best case complexity: can cheat
\item Average case complexity

\begin{itemize}
\item can depend on input distribution
\item randomized algorithm
\end{itemize}

\item Big-O notation

\end{itemize}

\end{document}